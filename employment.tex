\begin{rubric}{Employment History}
\subrubric{GE Vernova}
\entry*[2023 -- Present]%
	\textbf{Advanced Engineer, Conceptual Design and Aerodynamics}
 \begin{itemize}
     \item Owned Python/C++ blade conceptual design tools.
     \item Developed tools to automate the setup, job submission and post-processing for CFD on HPC.
     \item Developed a spline-based blade design tool in PyQt to manipulate blade shape parameters through control points. Implemented Bézier, Catmull-Rom and B-Splines.
 \end{itemize}
%
% Blank lines result in extra space!
%
\entry*[2021 -- 2023]%
	\textbf{Edison Engineering Development Program} (2 year technical leadership rotational program)
 \begin{itemize}
     \item Developed a steady state aeroelastic code in Python for wind turbine blades by coupling in-house aerodynamics (BEM) and structural (FEM, corotational beam elements) codes.
     \item Developed a FEM mesh manipulation library in C++, based on the Partial Entity data structure for representation of non-manifold edges.
     \item Contributed to the development of an aeroelastic mode tracking module in Python, based on a Modal Assurance Criterion (MAC).
     \item Implemented a critical Python module for improving 50-year extreme load extrapolation on the AWS simulation environment by making the tail of the load distribution denser.
\end{itemize}
\end{rubric}